% status:
% chapter: 

\title{Restful container for CompuCell3D}

\author{Juliano Ferrari Gianlupi}
\affiliation{%
  \institution{Indiana University}
  \country{USA}}
\email{jferrari@iu.edu}


% The default list of authors is too long for headers}
\renewcommand{\shortauthors}{J. F. Gianlupi}


\begin{abstract}

CompuCell3D is a Cellular Potts Model simulator. 

 
\end{abstract}

\keywords{hid-sp18-601, project, restful-compucell-3d}


\maketitle

\section{Introduction}\label{hid-sp18-601-project-section-introduction}

CompuCell3D uses Cellular Potts Model (CPM) to simulate systems containing 
cell 
structures~\cite{hid-sp18-601-paper-swat2012multi}, be them biological or not.
 In it cells are made up of voxels, each cell has an unique ID and have energies
 associated with it~\cite{hid-sp18-601-paper-glazier1993simulation}.

The biological system for which CPM was developed was the separation of hydra 
tissues~\cite{hid-sp18-601-paper-glazier1993simulation}. The energy Hamiltonian 
for this system had to have contact energy in order for the cells to separate. 
Since cells have a typical volume it also had to have an energy term related to 
the cells' volume, so that they would not  deviate much from it or disappear. 
The explicit Hamiltonian is shown in Equation 
\ref{hid-sp18-601-equation-cpm-orig-hamiltonian}.
\begin{equation}\label{hid-sp18-601-equation-cpm-orig-hamiltonian}
H = \sum_i \sum_{j_n} J(\tau_{\sigma(i)},\tau_{\sigma(j)
})(1 - \delta(\sigma_i,\sigma_j)) + \sum_{c}\lambda_{c}\left(V_{c
} - V^{TG}_{c}\right)^2\,\,.
\end{equation}
The first term in the Hamiltonian is the contact energy, the second volume
 energy. In the first 
term the double sum is made on cell pixels and that pixel neighborhoods, in here
the neighborhood is not necessarily only the first neighbors. In 
the second term the sum is made on cells. The energy magnitude is defined
by \textit{J} and $\lambda$.

CPM's Hamiltonian can be expanded with any kind 
of energy. Cellular Potts Model also can have a chemical field in a secondary 
pixel lattice.

The way the system evolves is by the Metropolis 
algorithm~\cite{hid-sp18-601-paper-metropolis1953equation}. This algorithm 
is based on attempting to change the pixel's owner.
 The first step is to randomly select one 
pixel and one of this pixel's first neighbors. If the pixels don't belong to 
the same cell an exchange attempt is made. The energy of the system is 
calculated before and after the pixel exchange. If the energy diminishes
the attempt is accepted, if it increases the exchange may still be accepted,
 depending on a probabilistic function.

\subsection{}\label{}



\bibliographystyle{ACM-Reference-Format}
\bibliography{project} 

