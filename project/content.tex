% status:
% chapter: 

\title{Restfull container for CompuCell3D}

\author{Juliano Ferrari Gianlupi}
\affiliation{%
  \institution{Indiana University}
  \country{USA}}
\email{jferrari@iu.edu}


% The default list of authors is too long for headers}
\renewcommand{\shortauthors}{J. F. Gianlupi}


\begin{abstract}

CompuCell3D is a Cellular Potts Model simulator. 

 
\end{abstract}

\keywords{hid-sp18-601, project, restfull-compucell-3d}


\maketitle

\section{Introduction}\label{hid-sp18-601-project-section-introduction}

CompuCell3D uses Cellular Potts Model (CPM) to simulate systems containing 
cell 
structures~\cite{hid-sp18-601-paper-swat2012multi}, be them biological or not.
 In it cells are made up of voxels, each cell has an unique ID and have energies
 associated with it~\cite{hid-sp18-601-paper-glazier1993simulation}.

The biological system for which CPM was developed was the separation of hydra 
tissues~\cite{hid-sp18-601-paper-glazier1993simulation}. The energy hamiltonian 
for this system had to have contact energy in order for the cells to separate. 
Since cells have a typical volume it also had to have an energy term related to 
the cells' volume, so that they would not  deviate much from it or disapear. 
It is
\begin{equation}\label{hid-sp18-601-equation-cpm-orig-hamiltonian}
H = \sum_i \sum_{j_v} J(\tau_{\sigma(i)},\tau_{\sigma(j)
})(1 - \delta(\sigma_i,\sigma_j)) + \sum_{c}\lambda_{c}\left(V_{c
} - V^{TG}_{c}\right)^2\,\,.
\end{equation}


\section{}\label{}



\subsection{}\label{}



\bibliographystyle{ACM-Reference-Format}
\bibliography{project} 

