\title{SETI$@$Home}

\author{Juliano Ferrari Gianlupi}
\affiliation{%
  \institution{Indiana University}
  \country{USA}}
\email{jferrari@iu.edu}


% The default list of authors is too long for headers}
\renewcommand{\shortauthors}{J. F. Gianlupi}


\begin{abstract}
After funding was cut SETI lauched SETI$@$Home, a public volunteer computer 
via the internet. Using this software users donate idle CPU time for SETI to do 
calculations~\cite{www-hid-sp18-601-sathome-about}. It was released in 1999 and 
one of its goals was to prove the viability of volunteer computing. This goal 
has succeded completly. SETI$@$Home was inspiration for several similar projects
 ~\cite{www-hid-sp18-601-boinc-projects}, one of each is the 
 LHC$@$home~\cite{www-hid-sp18-601-lhc-at-home-history}.
\end{abstract}

\keywords{hid-sp18-601, setiathome}


\maketitle

\section{Introduction}

\section{Background}

\section{How it works}

The data comes from the Arecibo radio telescope in Puerto Rico. As the facility
does not have high speed internet the data must first be send by regular mail to
 Berkeley~\cite{www-hid-sp18-601-sathome-howworks}. Once there the data is 
 diveded in "work units", which size is 250 kibibytes, and distributed to the 
 colaborator's computers on the network.

The data to be analized are radio waves. The frequency range of the waves 
analized in the SETI$@$Home project is from 1418MHz to 1421MHz organized in 
10Khz pieces~\cite{aper-hid-sp18-601-anderson2002seti}. A work unit is 100 
seconds of the recording of one of this 
slices~\cite{www-hid-sp18-601-sathome-howworks}.
 

\section{Best Practices}


\section{Conclusion}

\begin{acks}

\end{acks}

\bibliographystyle{ACM-Reference-Format}
\bibliography{report} 

